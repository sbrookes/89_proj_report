
% report for cs89 final project

\documentclass{article}

\usepackage{subcaption}
\usepackage{fullpage}
\usepackage{listings}
\usepackage{fixltx2e}
\usepackage{graphicx}
\usepackage{dirtree}
\usepackage{caption}
\usepackage{amsmath}
\usepackage{mdwlist}
\usepackage{natbib}
\usepackage{xcolor}
\usepackage{float}

\usepackage{hyperref}

\setlength{\parskip}{3mm}

\begin{document}

\title{Learning the Structure of Local Neural Circuits in Mouse Ectorhinal Cortex}
\author{Scott Brookes \& Derek Racine}

\maketitle

%%%%%%%%%%%%%%%%%%%%%%%%%%%%%%%%% INTRODUCTION %%%%%%%%%%%%%%%%%%%%%%%%%%%%%%%%
\section*{Introduction}
The brain is structured such that local networks of neurons often perform 
distinct, modular, tasks. For example, the ectorhinal cortex is known to 
receive the majority of its input from visual sensory areas and is involved 
with visual memory and object recognition. However, the network structure is 
not well understood. Additionally, the effect of training an animal to 
recognize specific images on the network has yet to be explored. In particular,
one might hypothesize that this training might cause the neurons in the area to
become tuned to images that the mouse has seen before, but not others. \par

%%%%%%%%%%%%%%%%%%%%%%%%%%%%%%%%%   DATASET    %%%%%%%%%%%%%%%%%%%%%%%%%%%%%%%%
\section*{Dataset}

We have obtained data sets corresponding to four recording sessions. The first 
data set consists of activity of neurons in the primary visual cortex that are 
well coordinated and known to respond to oriented gratings which are presented 
to the mice during the session, and thus will be a useful control for testing 
our methodology. The remaining data sets track the activity of neurons in the 
ectorhinal cortex during period of no stimulus, new stimulus, and known 
stimulus. Inferring both network structure and edge weights will allow us to 
determine how the neurons react to new stimulus, and whether or not the 
neurons become tuned to stimuli that the mouse has learned. \par

%%%%%%%%%%%%%%%%%%%%%%%%%%%%%%%%%    METHOD    %%%%%%%%%%%%%%%%%%%%%%%%%%%%%%%%
\section*{Method}

We propose to model the local neural circuits as a Bayesian network and infer 
the edges from the correlations of the activity of individual neurons, which 
we will treat as binary random variables – firing or not firing. In order to 
learn the structure, we plan to discretize our continuous series into buckets 
that each represent a single instance of the overall network. \par

%%%%%%%%%%%%%%%%%%%%%%%%%%%%%%%%%  EVALUATION  %%%%%%%%%%%%%%%%%%%%%%%%%%%%%%%%
\section*{Measuring Success}

We expect that the model we learn on the first data set of highly correlated 
activity in the primary visual cortex will align well with the theoretical 
expectation. To clarify, we plan on using the same methods that we test on 
this first data set in order to verify their correctness/applicability, on the 
others. \par

\end{document}
