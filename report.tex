
% report for cs89 final project

\documentclass{article}

\usepackage{subcaption}
\usepackage{fullpage}
\usepackage{listings}
\usepackage{fixltx2e}
\usepackage{graphicx}
\usepackage{dirtree}
\usepackage{caption}
\usepackage{amsmath}
\usepackage{mdwlist}
\usepackage{natbib}
\usepackage{xcolor}
\usepackage{float}

\usepackage{hyperref}

\setlength{\parskip}{3mm}

\begin{document}

\title{Learning the Structure of Local Neural Circuits in Mouse Ectorhinal Cortex}
\author{Scott Brookes and Derek Racine}

\maketitle

%%%%%%%%%%%%%%%%%%%%%%%%%%%%%%%%% INTRODUCTION %%%%%%%%%%%%%%%%%%%%%%%%%%%%%%%%
\section*{Introduction}
The brain is structured such that local networks of neurons can perform 
distinct, modular, tasks. For example, the ectorhinal cortex is known to 
receive the majority of its input from visual sensory areas and is involved 
in visual memory and object recognition. However, the network structure is 
not well understood. Training an animal to recognize certain images, such as 
through visual paired-associate learning, might also affect the network 
dynamics. In particular, one might hypothesize that this training could cause 
neurons to become tuned to images that the mouse has seen before, but not 
others. That is, the activity of neurons would be well correlated with 
stimuli used during training. Yet, this relationship hasn't been 
explored. \par 

Simplified neurons can be described as being in one of two states: resting or 
firing. When a neuron fires, calcium ions flow into it from the extracellular 
fluid. As such, fluorescence observed from calcium sensors can serve as a 
proxy for neural activity. \par

The recent discovery of an ultrasensitive family of fluorescent calcium 
sensors has caused an explosion in the availability of time-series data based 
on the use of this technique.\cite{chen13} Whereas previous recording methods 
were limited to only a few cells, fluorescent imaging techniques can measure 
the activity of an entire region of neurons simultaneously. This data offers 
an opportunity to learn the structure of local neural circuits through 
statistical methods. Indeed, previous work has addressed this problem by 
modeling fluorescent imaging data as a collection of coupled Hidden Markov 
chains, one for each neuron.\cite{mishchenko11} \par

%%%%%%%%%%%%%%%%%%%%%%%%%%%%%%%%%   DATASET    %%%%%%%%%%%%%%%%%%%%%%%%%%%%%%%%
\section*{Dataset}

We have obtained data sets corresponding to four recording sessions. The first 
data set consists of activity of neurons in the primary visual cortex that are 
well coordinated and known to respond to oriented gratings which are presented 
to the mice during the session, and thus will be a useful control for testing 
our methodology. The remaining data sets track the activity of neurons in the 
ectorhinal cortex during period of no stimulus, new stimulus, and known 
stimulus. Inferring both network structure and edge weights will allow us to 
determine how the neurons react to new stimulus, and whether or not the 
neurons become tuned to stimuli that the mouse has learned. \par

%%%%%%%%%%%%%%%%%%%%%%%%%%%%%%%%%    METHOD    %%%%%%%%%%%%%%%%%%%%%%%%%%%%%%%%
\section*{Method}

We propose to model the local neural circuits as a Bayesian network and infer 
the edges from the correlations of the activity of individual neurons, which 
we will treat as binary random variables – firing or not firing. In order to 
learn the structure, we plan to discretize our continuous series into buckets 
that each represent a single instance of the overall network. \par

%%%%%%%%%%%%%%%%%%%%%%%%%%%%%%%%%  EVALUATION  %%%%%%%%%%%%%%%%%%%%%%%%%%%%%%%%
\section*{Measuring Success}

We expect that the model we learn on the first data set of highly correlated 
activity in the primary visual cortex will align well with the theoretical 
expectation. To clarify, we plan on using the same methods that we test on 
this first data set in order to verify their correctness/applicability, on the 
others. \par

\bibliographystyle{plain}
\bibliography{bibliography}

%\section*{References}
%[1] Mishchenko, Yuriy, Joshua T. Vogelstein, and Liam Paninski. "A Bayesian approach for inferring neuronal connectivity from calcium fluorescent imaging data." The Annals of Applied Statistics 5.2B (2011): 1229-1261.

%[2] Chen, Tsai-Wen, et al. "Ultrasensitive fluorescent proteins for imaging neuronal activity." Nature 499.7458 (2013): 295-300.

\end{document}
